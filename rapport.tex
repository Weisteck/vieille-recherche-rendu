\documentclass[11pt,a4paper,oneside]{book}
\usepackage{fontspec}
\usepackage[top=2.5cm, bottom=2.5cm, left=2.5cm, right=2.5cm]{geometry}
\usepackage[utf8]{inputenc}
\usepackage[T1]{fontenc}
\usepackage[francais]{babel}
\usepackage{color}
\usepackage[table]{xcolor} %couleur dans un tableau
\usepackage{multirow} %fusion des lignes d'un tableau
\usepackage{tabularx}
\usepackage{pdflscape}


% ceci est un commentaire
\author{Hugo CELDRAN}
\title{Vieille et Recherche}
\title{Expert Cloud, Sécurité et Infrastructure}
\date{2022}

\setmainfont{Calibri-Regular.ttf}[Path=./calibri/,
BoldFont       = Calibri-Bold.TTF,
BoldItalicFont = Calibri-Bold-Italic.ttf,
ItalicFont     = Calibri-Italic.ttf]

\renewcommand{\baselinestretch}{1.5}

\begin{document}

\maketitle
\tableofcontents


\chapter{Introduction}
% dire que je fais que de la veille cible, veille radar twitter

Dans le cadre de mon alternance, j'ai pris la décision de changer d'entreprise pour réaliser mon master.
En effet, ce choix était motivé par la volonté de découvrir un nouveau métier et les nouvelles missions qui l'accompagnent. \\
J'ai découvert alors le métier d'administrateur système dans une culture DevOps.
Le DevOps est un mouvement en ingénierie informatique et une pratique technique qui veut unifier le développement logiciel (Dev) et l'administration des infrastructures informatiques (Ops).
Le problème avec cette approche c'est qu'il n'y a pas d'aspect sécurité des systèmes et des logiciels.
Donc vous aviez des processus de développement et de déploiement confié à une équipe et à la phase finale une autre équipe gérée la sécurité.
Cela n'était pas gênant à une époque car les déploiement applicatifs ne se faisaient pas aussi régulièrement qu'aujourd'hui.
Par exemple, où je travaille, nous faisons environ cinq de déploiement par jour. Alors imaginons nous des grandes entreprises qui travaillent selon cette culture : Google, Netflix. Le nombre de déploiement est énorme alors ne pas inclure de sécurité dedans serait une erreur. \\
En effet dans le cadre de travail collaboratif du modèle DevOps, la sécurité est une responsabilité partagée, intégrée du début à la fin. Cette notion est si importante qu'elle a donné naissance à l'expression « DevSecOps » pour souligner la nécessité d'intégrer la sécurité aux projets DevOps. \\
C'est donc dans cette démarche que l'on peut se poser la problématique suivante : En quoi le DevSecOps contribue t il à améliorer l'infrastructure et son déploiement ? \\
Cette vieille a donc une importance dans mon projet professionnel et mon quotidien. \\
Nous verrons dans un premier la démarche méthodologique afin d'expliquer le procédé de l'élaboration de mon plan de veille, puis dans un second temps la présentation des résultats obtenues en trois axes : veille concurrentielle, technique / technologique et commerciale.


\chapter{Démarche méthodologique}

Dans cette partie, nous verrons comment j'ai procédé à l'élaboration de mon plan de veille, puis des mots clefs que j'ai pu utiliser selon mon sujet du DevSecOps ainsi que le tableau des ressources immobilisées.

\section{Élaboration du plan de veille}

Dans cette section, nous allons voir en 5 points ce qui m'a permit de réaliser mon plan de veille :
\begin{enumerate}
\item Identifier mon besoin ainsi que définir le périmètre qui en découle,
\item Collecter les informations qui m'intéressent,
\item Qualifier afin de savoir quelle information je compte conserver,
\item Organiser pour savoir si les informations collecter son pertinente,
\item Partager et utiliser afin de ne pas être seul à me servir de cette veille.
\end{enumerate}


\subsection{Définir le périmètre}

Mon sujet sur le DevSecOps est venu naturellement pour répondre à un besoin quotidien, mon métier.
En effet, cette veille me permet de me tenir au courant de toutes les mises à jours qui peuvent sortir ou encore me tenir au courant des différentes failles de sécurité qu'il peut exister pour m'en prémunir. \\
Cette veille est utile dans mon métier d'aujourd'hui mais ne correspond pas forcément à d'autres besoins à d'autres postes ou encore dans une entreprise différente. \\
Il a fallu que je définisses un périmètre afin d'être efficace et efficient. \\
Le périmètre peut être dans les sujets cherchés ou dans les ressources abordés.
Cela permet de ne pas perdre son temps sur internet en s'écartant du sujet et des recherches que l'on nous voulons faire. 
Les recherches sont souvent en anglais, tous les sites officiels des distrubuteurs des logiciels et donc les documentations le sont, et en francais pour des articles détaillés sur les LinuxMagazine en général. \\
Je fais ma veille une à deux heures par jour en jours ouvrés.

\subsection{Collecter}

Pour collecter les informations de veille, j'ai utilisé un flux RSS.
Un flux RSS permet de récupérer sous format xml une information de mise à jour d'un site internet. Cela permet donc de surveiller plusieurs sources internet de façon automatiser et d'un seul et même endroit, notre outil choisi de flux RSS. \\
Pour moi, la mise en place de cet outil était pertinent. Car pour toute la veille des mise à jours des plusieurs micro logiciels dont on se sert dans l'entreprise, cela devient vite chronophage si l'information ne vient pas à nous. \\
Cela permet auss d'éviter de consulter énormément de sites pour avoir les dernières informations qui sortent.

\subsection{Qualifier}

Une fois que l'information est venue jusqu'à nous, il faut à présent la qualifié, cela signifie que nous devons la classer.
Par exemple :
\begin{itemize}
\item Si une nouvelle technologie arrive bientôt, nous voudrons surveiller son évolution pour la tester et pourquoi pas la mettre en place à l'avenir,
\item Si par contre c'est une information d'une nouvelle mise à jour qui corrige un problème de  sécurité sur un logiciel il faudra rapidement le mettre en place.
\end{itemize}
Cette étape me permet de voir si les flux RSS reçus sont pertinents, car certains flux peuvent devenir obsolète (le logiciel suivi n'est plus utilisé) ou plus pertinent, je me suis abonné à un site qui ne publie pas que des articles sur mon domaine. \\
Dans certains cas, il faudra supprimer la source et d'entre cas réadapter le flux reçu.

\subsection{Organiser}

Une fois l'information qualifié, il faudra l'organiser, c'est à dire le prioritisé en fonction de là où provient l'information. \\
Par exemple, si une nouvelle mise à jour d'un logiciel est sorti comme cela vient des sources officielles du concepteur je sais que je dois le priotiser en premier par rapport à une veille radar avec twitter ou beaucoup d'informations superflux peuvent remonter. \\
Ce travail peut également se faire en configurant son flux dans des catégories.
Mais si un site est plus général, comme par exemple ma veille radar que je fais sur twitter, on pourra le faire après avoir fait la qualification.
En effet, cela permettra de vérifier les sources est donc de déterminer si l'information est fiable et utilisable par la suite.

\subsection{Partager et utiliser}

J'ai choisi pour faire ma veille un outil de flux RSS (Tiny Tiny RSS) que l'on peut soi-même héberger.
On peut donc créer plusieurs utilisateurs qui interviendront sur la veille, mettre des catégories communes. \\
L'accès est fait grâce à une page internet, où l'on doit ensuite rentrer son nom d'utilisateur et son mot de passe. Mais cela reste sécurisé et seulement exploitable par les membres de l'entreprise. \\

\newpage

\section{Listes des mots clés}

Ces mots clés sont les mots principaux que j'utilise pour ma veille radar. \\

\begin{tabularx}{14cm}{|X|}
\hline
\rowcolor{gray!30} Mots clefs \\
\hline
DevOps \\
DevSecOps \\
Linux \\
Terraform \\
Hashicorps \\
Kubernetes \\
Containers \\
Docker \\
Azure \\
AWS \\
Cloud \\
CNCF \\
Cloud Native Interactive Landscape \\
Helm \\
Gitlab \\
Deployement \\
Automatization \\
CI \\
CD \\
Infrastructure as code \\
CVE \\
Common Vulnerabilities and Exposures \\
\hline
\end{tabularx}

\newpage

\begin{landscape}

\section{Tableau des ressources immobilisées}

Nous allons voir dans le tableau suivant les ressources immobilisées dans mon outil de flux RSS, je ne présente que les principales et les plus pertinents que j'utilise pour cette veille. \\

\begin{tabular}{|l|c|c|c|c|c|c|}
\hline
\rowcolor{gray!30} Nom de la source / d'outil & Type & Lien & Auteur & Fréquence de la veille & Mots Clés & Provenance \\
\hline
GitHub Notifications release & technique & Privé & Multiple & tous les jours & & Flux RSS \\
CVE (faille de sécurité) & technique & https://cve.report/ & CVE report & tous les jours & & Flux RSS \\
Twitter Search & concurrentiel & https://twitter.com & Multiple & tous les jours & DevOps, DevSecOps & Réseau social \\
\hline
\end{tabular}

% Rajouter dans ce tableau les conferences types KubeCon & Mixit & JDLL

\end{landscape}


\chapter{Corps du dossier}

\section{Le DevSecOps}

\subsection{Définition}

\subsection{Historique}

\subsection{Importance dans la société actuelle}

\newpage

\section{Le Marché}

\subsection{État du marché}

\subsection{Principaux acteurs}

\subsection{Parties prenantes}

\newpage

\section{Les ressources}

\subsection{Technologies identifiées}

\subsection{Innovations et tendances}

\subsection{Évolution potentielles}

\newpage

\section{Les enjeux et les contreverses}

\subsection{Économiques}

\subsection{Stratégiques}


\chapter{Conclusion}

% Expliquer que les methodes choisies on etait faite en fonction du sujet choisi
% Analyse réflexive sur votre démarche de veille et prise du recul des connaissances acquises

\chapter{Bibliographie}

% mettre les articles qui m'ont aide a faire le rapport

\end{document}
